\chapter{機率與機率分布}

\section{期望值與變異數}

\section{機率質量與離散型分布}

\subsection{白努利分布(Bernoulli)}

\subsection{二項分布(Binomial)}

\subsection{卜瓦松分布(Poisson)}

\subsection{超幾何分布(Hypergeometric)}

\section{機率密度與連續型分布}
在前面的討論中,我們都假設隨機變數的可能取值可以用手指頭一個一個地數出來,在數學上被稱為「離散(discrete)」。
例如白努利分布的取值僅可能是$0$或$1$;
二項式分布的取值可能是$0, 1, 2, 3, ..., n$;
卜瓦松分布的取值可能是$0, 1, 2, 3...$。那麼,如果隨機變數的可能取值是連續的(continuous),機率的計算方法會不會有變化?我們思考以下情況:

\smallskip

\begin{custom}{思考}
假設我們從$0$到$1$的區間隨機抽一個實數,那麼抽到$0.5$的機率是多少?
\end{custom}

\smallskip

有那麼多個實數可以抽,剛好抽到$0.5$的機率應該很低吧?很不幸地,不論我們把機率設得多低,終究都會違反「機率總和等於一」的大前提。舉例來說,如果設定抽到$0.5$的機率是$10^{-12}$,那麼總是能找到$10^{13}$個其他可能取值,它們的出現機率也應該都是$10^{-12}$。這些取值出現的機率總和是$10^{13}\times10^{-12}=10$,超過$1$了!因此,我們得到一個奇怪的結論:\textbf{若從$0$到$1$的區間隨機取一個實數,則抽到$0.5$的機率是$0$}。

這實在太詭異了!「抽到$0.5$」這個事件明明就可能發生,但是它的發生機率卻是$0$?在$0$到$1$之間所有實數抽到的機率都是$0$,但是總和起來的機率卻是$1$?因此,我們在離散隨機變數中「機率等於零就是不會發生」的機率概念,不能套用在連續隨機變數中。在連續隨機變數的情況下,「出現機率為零」不一定代表「不會發生」,在數學上被稱為「almost never(幾乎從不發生)」。

如果每個取值的機率都是$0$,我們要怎麼描述連續隨機變數的機率分布呢?
此處我們必須借助區間的機率。
例如從$0$到$1$的區間隨機抽一個實數,則此實數介於$0.5$到$0.6$之間的機率應該是$0.1$。
利用區間機率,我們就可以了解機率密度(probability density)的概念了。
下面這個表是標準常態分布在$0$附近的區間機率:

\begin{tabular}{cccccccc}
    區間 & (-0.5,0.5) & (-0.1,0.1) & (-0.05,0.05) & (-0.01,0.01) & (-0.0001,0.0001)\\
    機率 & 0.3829 & 0.0797 & 0.03988 & 0.007979 & 0.00007979\\
    區間寬度 & 1 & 0.2 & 0.1 & 0.02 & 0.0002\\
    $\frac{\text{機率}}{\text{區間寬度}}$ & 0.3829 & 0.3983 & 0.3988 & 0.3989 & 0.3989
\end{tabular}

\subsection{均一分布(Uniform)}

\subsection{常態分布(Normal)}

\subsection{卡方分布(Chi-squared)}

\subsection{$t$分布($t$)}

$t$分布可說是常態分布的兄弟,它有一個參數:自由度$\nu$

\begin{bonus}{$t$分布的定義與推導}{tdist}
自由度為$\nu$的$t$分布來自於標準常態分布與自由度為$\nu$的卡方分布。完整定義如下:
$$\left\{\begin{aligned}
U &\sim N(0,1)\\
V &\sim \chi^2(\nu)
\end{aligned}\right. ;\; U \bot V \; \Rightarrow \; \frac{U}{V/\nu} \sim t(\nu)$$
其機率密度函數為
$$f_t(x;\nu)=\frac{\Gamma(\frac{\nu+1}{2})}{\sqrt{\nu\pi}\Gamma(\frac{\nu}{2})}\left(1+\frac{t^2}{\nu}\right)^{-\frac{\nu+1}{2}}$$
\end{bonus}

\section{大數法則}

\section{中央極限定理}

\section{條件機率與貝式定理}